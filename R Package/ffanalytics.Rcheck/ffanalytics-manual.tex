\nonstopmode{}
\documentclass[a4paper]{book}
\usepackage[times,inconsolata,hyper]{Rd}
\usepackage{makeidx}
\usepackage[utf8,latin1]{inputenc}
% \usepackage{graphicx} % @USE GRAPHICX@
\makeindex{}
\begin{document}
\chapter*{}
\begin{center}
{\textbf{\huge Package `ffanalytics'}}
\par\bigskip{\large \today}
\end{center}
\begin{description}
\raggedright{}
\item[Type]\AsIs{Package}
\item[Title]\AsIs{Fantasy Football Analytics}
\item[Version]\AsIs{0.1.91}
\item[Author]\AsIs{Dennis Andersen [aut, cre], Isaac Petersen [ctb]}
\item[Maintainer]\AsIs{Dennis Andersen }\email{andersen.dennis@live.com}\AsIs{}
\item[Description]\AsIs{Scrape fantasy football projections from multiple sources and
calculate projected fantasy points for your drafts.}
\item[License]\AsIs{GPL-3}
\item[URL]\AsIs{}\url{http://fantasyfootballanalytics.net/}\AsIs{}
\item[LazyData]\AsIs{TRUE}
\item[Depends]\AsIs{shiny, miniUI}
\item[Imports]\AsIs{data.table, stringr, DT, XML, httr, tcltk, RCurl, Hmisc,
readxl, RSelenium}
\item[RoxygenNote]\AsIs{5.0.1}
\item[Collate]\AsIs{'calcStdDev.R' 'calculatePoints.R' 'calculateRisk.R'
'calculationDataPrep.R' 'columnVariables.R'
'confidenceInterval.R' 'createObject.R' 'data.R' 'dataGadget.R'
'dataPeriod.R' 'dataResult.R' 'externalConstants.R'
'extractTeamNames.R' 'ffanalytics.R' 'getAdpAav.R'
'getAdpData.R' 'getPlayerData.R' 'getPlayerName.R'
'getProjections.R' 'getRanks.R' 'sourceSite.R'
'sourceAnalyst.R' 'sourceTable.R' 'getUrls.R'
'helperFunctions.R' 'internalConstants.R' 'leagueScoring.R'
'mflPlayers.R' 'nflPlayerData.R' 'projectPoints.R'
'projectionGadget.R' 'readUrl.R' 'retrieveData.R' 'runScrape.R'
'scoringUI.R' 'scrapeGadget.R' 'scrapeJQuery.R'
'scrapeJSONdata.R' 'tableMerge.R' 'tierFunction.R'
'yahooBasics.R'}
\item[NeedsCompilation]\AsIs{no}
\end{description}
\Rdcontents{\R{} topics documented:}
\inputencoding{utf8}
\HeaderA{analystOptions}{Analyst options for a period}{analystOptions}
%
\begin{Description}\relax
Find the analysts that are projecting stats for the provided period
\end{Description}
%
\begin{Usage}
\begin{verbatim}
analystOptions(period)
\end{verbatim}
\end{Usage}
\inputencoding{utf8}
\HeaderA{analystPositions}{Analyst positions}{analystPositions}
\keyword{datasets}{analystPositions}
%
\begin{Description}\relax
Data table identifying which positions the different analysts are projecting
for
\end{Description}
%
\begin{Usage}
\begin{verbatim}
analystPositions
\end{verbatim}
\end{Usage}
%
\begin{Format}
A \LinkA{data.table}{data.table} with 151 rows and 4 columns
\begin{description}

\item[analystId] An integer identifying the analyst the row is referring to.
Refers to the linkanalyst table.
\item[position] Name of the position
\item[season] Indicates whether the analysts projects seasonal data for the
position.
\item[weekly] Indicates whether the analysts projects weekly data for the
position.

\end{description}
\end{Format}
\inputencoding{utf8}
\HeaderA{analysts}{Analyst data}{analysts}
\keyword{datasets}{analysts}
%
\begin{Description}\relax
Data table with information on analysts that are bing used for data scrapes.
\end{Description}
%
\begin{Usage}
\begin{verbatim}
analysts
\end{verbatim}
\end{Usage}
%
\begin{Format}
A \LinkA{data.table}{data.table} with 27 rows and 7 columns
\begin{description}

\item[analystId] Unique integer identifier for the analyst
\item[analystName] Name of analyst
\item[siteId] An integer identifying the site that the analyst is projecting
for. Refers to \LinkA{sites}{sites} table.
\item[season] Indicator of whether the analyst projects seasonal data or not
\item[weekly] Indicator of whether the analyst projects weekly data or not
\item[sourceId] The identifier that the site uses for the analyst if any
\item[weight] The weight used for the analyst in the weighted calculations

\end{description}
\end{Format}
\inputencoding{utf8}
\HeaderA{avgValue}{Calculate average based on selected method}{avgValue}
%
\begin{Description}\relax
Will calculate average of provided data and weights based on the selected
method
\end{Description}
%
\begin{Usage}
\begin{verbatim}
avgValue(calcMethod = "weighted", dataValue = as.numeric(),
  dataWeights = as.numeric(), na.rm = FALSE)
\end{verbatim}
\end{Usage}
%
\begin{Arguments}
\begin{ldescription}
\item[\code{calcMethod}] One of \code{c("average", "weighted", "robust")}

\item[\code{dataValue}] A numeric vector of values to average base on chosen calculation
method.

\item[\code{dataWeights}] A numeric vector of weight values associated with the
dataValue parameter.

\item[\code{na.rm}] A logical value determining if NA values should be removed.
\end{ldescription}
\end{Arguments}
\inputencoding{utf8}
\HeaderA{calcStdDev}{Calculate Standard Deviation}{calcStdDev}
%
\begin{Description}\relax
Standard Deviation is calculated based on method passed to the function.
\end{Description}
%
\begin{Usage}
\begin{verbatim}
calcStdDev(calcMethod = "weighted", dataValue = as.numeric(),
  dataWeights = as.numeric(), na.rm = FALSE)
\end{verbatim}
\end{Usage}
%
\begin{Arguments}
\begin{ldescription}
\item[\code{calcMethod}] Calculation method. Can be one of "weighted", "robust" or
"average"

\item[\code{dataValue}] Vector of values to calculate standard deviation for

\item[\code{dataWeights}] Vector of weights for weighted calculation
\end{ldescription}
\end{Arguments}
\inputencoding{utf8}
\HeaderA{calculatePoints}{Caluclate fantasy points}{calculatePoints}
%
\begin{Description}\relax
Based on list of scoring rules calculate fantasy points using projection data
passed in a table
\end{Description}
%
\begin{Usage}
\begin{verbatim}
calculatePoints(projectionData = data.table(), scoringRules = list())
\end{verbatim}
\end{Usage}
%
\begin{Arguments}
\begin{ldescription}
\item[\code{projectionData}] A \LinkA{data.table}{data.table} with projected stats.
Should from the \LinkA{getMeltedData}{getMeltedData} function

\item[\code{scoringRules}] A list of scoring rules with one element per position
\end{ldescription}
\end{Arguments}
%
\begin{Value}
A \LinkA{data.table}{data.table} with up to 4 cloumns
\begin{description}

\item[position] The position the player is playing
\item[playerId] The ID of the player. Get player names by merging results from
\LinkA{getPlayerData}{getPlayerData}
\item[analyst] The ID of the analyst which projections are used as the basis
for the points
\item[points] The calculated number of fantasy points

\end{description}

\end{Value}
\inputencoding{utf8}
\HeaderA{calculateRisk}{Risk calculation based on to variables}{calculateRisk}
%
\begin{Description}\relax
Calculation of risk is done by scaling the standard deviation variables
passed and averaging them before returning a measure with mean 5 and standard
deviation of 2
\end{Description}
%
\begin{Usage}
\begin{verbatim}
calculateRisk(var1, var2)
\end{verbatim}
\end{Usage}
\inputencoding{utf8}
\HeaderA{calculateVor}{Calculate Value Over Replacement}{calculateVor}
%
\begin{Description}\relax
Based on provided ranks and points calculate the value over replacement.
Function uses \LinkA{vorBaseline}{vorBaseline} and \LinkA{vorAdjustment}{vorAdjustment} in the calculation.
Please adjust these to match your league before running calculation.
\end{Description}
%
\begin{Usage}
\begin{verbatim}
calculateVor(ranks, points, position)
\end{verbatim}
\end{Usage}
%
\begin{Arguments}
\begin{ldescription}
\item[\code{ranks}] Player ranks

\item[\code{points}] Player Points

\item[\code{position}] Player Position. Used to extract value from \LinkA{vorBaseline}{vorBaseline}
and \LinkA{vorAdjustment}{vorAdjustment}
\end{ldescription}
\end{Arguments}
\inputencoding{utf8}
\HeaderA{clusterTier}{Set tiers based on clusters}{clusterTier}
%
\begin{Description}\relax
Set tiers based on clusters
\end{Description}
%
\begin{Usage}
\begin{verbatim}
clusterTier(points, position)
\end{verbatim}
\end{Usage}
\inputencoding{utf8}
\HeaderA{confidenceInterval}{Calculate confidence interval}{confidenceInterval}
%
\begin{Description}\relax
Confidence intervals are determined as percentiles (default 10 and 90).
If the correcpsonding average value is less than the low point then we
set the lower point to the minimum value. Conversely if the average value is
higher than the upper limit then we set the upper limit to the max value.
\end{Description}
%
\begin{Usage}
\begin{verbatim}
confidenceInterval(calcMethod = "weighted", dataValue = as.numeric(),
  dataWeights = as.numeric(), pValues = c(0.1, 0.9), na.rm = FALSE)
\end{verbatim}
\end{Usage}
%
\begin{Arguments}
\begin{ldescription}
\item[\code{calcMethod}] Calculation method. Can be one of "weighted", "robust" or
"average"

\item[\code{dataValue}] Vector of values to calculate confidence interval for

\item[\code{dataWeights}] Vector of weights for weighted calculation

\item[\code{pValues}] Vector of percentiles for the confidence interval. Defaults to
\code{c(0.1, 0.9)} for 10 and 90 percentiles.
\end{ldescription}
\end{Arguments}
\inputencoding{utf8}
\HeaderA{Constants}{Package Constants}{Constants}
\aliasA{position.Id}{Constants}{position.Id}
\aliasA{position.name}{Constants}{position.name}
\aliasA{yahooLeague}{Constants}{yahooLeague}
\keyword{datasets}{Constants}
%
\begin{Description}\relax
Constants build into the package
\end{Description}
%
\begin{Usage}
\begin{verbatim}
position.name

position.Id

yahooLeague
\end{verbatim}
\end{Usage}
%
\begin{Details}\relax
A number of constants are included in the package
\end{Details}
\inputencoding{utf8}
\HeaderA{createObject}{Create objects from list}{createObject}
%
\begin{Description}\relax
Allows creation of class objects based on a list.
\end{Description}
%
\begin{Usage}
\begin{verbatim}
createObject(objName = as.character(), targetValues = list())
\end{verbatim}
\end{Usage}
%
\begin{Arguments}
\begin{ldescription}
\item[\code{objName}] Name of object to be created

\item[\code{targetValues}] List of values representing slots in the object
\end{ldescription}
\end{Arguments}
\inputencoding{utf8}
\HeaderA{dataGadget}{Gadget used to display results of scrape calculations}{dataGadget}
%
\begin{Description}\relax
Gadget used to display results of scrape calculations
\end{Description}
%
\begin{Usage}
\begin{verbatim}
dataGadget(inputData)
\end{verbatim}
\end{Usage}
\inputencoding{utf8}
\HeaderA{dataPeriod-class}{Class to represent the period for a datascrape}{dataPeriod.Rdash.class}
\aliasA{dataPeriod}{dataPeriod-class}{dataPeriod}
%
\begin{Description}\relax
The datascrapes need a week and season designation to work.
\end{Description}
%
\begin{Section}{Slots}

\begin{description}

\item[\code{weekNo}] An integer representing the week number. 0 for preseason, 1-17
for regular season and 18-21 for post season

\item[\code{season}] A year representing the season. Should be 2008 or later but
can't be higher than current year

\end{description}
\end{Section}
%
\begin{Examples}
\begin{ExampleCode}
dataPeriod(weekNo = 1, season = 2015) # Week 1 of the 2015 season
dataPeriod(season = 2015)             # 2015 season
dataPeriod(weekNo = 3)                # Week 3 of the current year
dataPeriod()                          # Current season
\end{ExampleCode}
\end{Examples}
\inputencoding{utf8}
\HeaderA{dataResult-class}{Class to represent scrape results}{dataResult.Rdash.class}
\aliasA{dataResult}{dataResult-class}{dataResult}
%
\begin{Description}\relax
Class to represent scrape results
\end{Description}
%
\begin{Section}{Slots}

\begin{description}

\item[\code{position}] Position that the scrape data isfor

\item[\code{resultData}] A \LinkA{data.table}{data.table} holding the data. The columns in the
table are determined by the \LinkA{staticColumns}{staticColumns} and \LinkA{resultColumns}{resultColumns}.

\end{description}
\end{Section}
\inputencoding{utf8}
\HeaderA{dropoffValue}{Calculate Dropoff value}{dropoffValue}
%
\begin{Description}\relax
Calculate Dropoff value
\end{Description}
%
\begin{Usage}
\begin{verbatim}
dropoffValue(dataValue)
\end{verbatim}
\end{Usage}
\inputencoding{utf8}
\HeaderA{ffa.scoreThreshold}{Default Scoring threshold for tiers}{ffa.scoreThreshold}
\keyword{datasets}{ffa.scoreThreshold}
%
\begin{Description}\relax
Default Scoring threshold for tiers
\end{Description}
%
\begin{Usage}
\begin{verbatim}
ffa.scoreThreshold
\end{verbatim}
\end{Usage}
%
\begin{Format}
An object of class \code{numeric} of length 9.
\end{Format}
\inputencoding{utf8}
\HeaderA{ffa.tierGroups}{Default number of tiers for clusters}{ffa.tierGroups}
\keyword{datasets}{ffa.tierGroups}
%
\begin{Description}\relax
Default number of tiers for clusters
\end{Description}
%
\begin{Usage}
\begin{verbatim}
ffa.tierGroups
\end{verbatim}
\end{Usage}
%
\begin{Format}
An object of class \code{numeric} of length 9.
\end{Format}
\inputencoding{utf8}
\HeaderA{ffa.vorAdjustment}{Default VOR Adjustments}{ffa.vorAdjustment}
\keyword{datasets}{ffa.vorAdjustment}
%
\begin{Description}\relax
Default VOR Adjustments
\end{Description}
%
\begin{Usage}
\begin{verbatim}
ffa.vorAdjustment
\end{verbatim}
\end{Usage}
%
\begin{Format}
An object of class \code{numeric} of length 9.
\end{Format}
\inputencoding{utf8}
\HeaderA{ffa.vorBaseline}{Default VOR Baseline}{ffa.vorBaseline}
\keyword{datasets}{ffa.vorBaseline}
%
\begin{Description}\relax
Default VOR Baseline
\end{Description}
%
\begin{Usage}
\begin{verbatim}
ffa.vorBaseline
\end{verbatim}
\end{Usage}
%
\begin{Format}
An object of class \code{numeric} of length 9.
\end{Format}
\inputencoding{utf8}
\HeaderA{ffanalytics}{Scraping and calculating data to use for fantasy football projections.}{ffanalytics}
\aliasA{ffanalytics-package}{ffanalytics}{ffanalytics.Rdash.package}
%
\begin{Description}\relax
The ffanalytics package provides three categories of important functions:
scrape, calculation and analysis.
\end{Description}
%
\begin{Section}{Scrape functions}

The scrape flow works like this:
\begin{enumerate}

\item{} User initiates the script and specifies the data period that needs to
be scraped
\item{} The scripts displays available analysts to scrape and the user selects
which to use
\item{} The script then displays available positions and asks the user to
select positions to scrape.
\item{}  Data scrape is executed and returns a list with a data table for each
position
\end{enumerate}

User can next specify which aggregate method to apply and execute the
calculation scripts on this list to get a data table with projected points,
confidence intervals, rankings, risk etc.

Tiers are calculated using effect size thresholds based on Cohen's d.
D value thresholds for determining tiers for each position can be set by:
\code{tierDValues <- c(QB = 0.25, RB = 0.4, WR = 0.4, TE = 0.35, K = 0.15, DST = 0.1, DL = 0.3, DB = 0.13, LB = 0.3)}
\end{Section}
\inputencoding{utf8}
\HeaderA{firstLast}{Reverse Last and First Name}{firstLast}
%
\begin{Description}\relax
Takes an input string in the form of "Last Name, First Name" and converts it
to "First Name Last Name".
\end{Description}
%
\begin{Usage}
\begin{verbatim}
firstLast(lastFirst)
\end{verbatim}
\end{Usage}
%
\begin{Arguments}
\begin{ldescription}
\item[\code{lastFist}] A string with the name to be converted
\end{ldescription}
\end{Arguments}
%
\begin{Examples}
\begin{ExampleCode}
firstLast("Smith, John") # Will return John Smith
\end{ExampleCode}
\end{Examples}
\inputencoding{utf8}
\HeaderA{getAAVdata}{Get AAV data}{getAAVdata}
%
\begin{Description}\relax
Retrieve AAV data from multiple sources and combine into average.
\end{Description}
%
\begin{Usage}
\begin{verbatim}
getAAVdata(AAVsources = c("ESPN", "MFL", "NFL"),
  season = as.POSIXlt(Sys.Date())$year + 1900, teams = 12,
  format = "standard", mflMocks = NULL, mflLeagues = NULL)
\end{verbatim}
\end{Usage}
%
\begin{Arguments}
\begin{ldescription}
\item[\code{season}] The season the ADP data is from

\item[\code{teams}] Number of teams in the league

\item[\code{format}] The format of the league, i.e. standard, ppr

\item[\code{mflMocks}] Include mock drafts from MFL. Set to 1 if only mock drafts
should be used, 0 if only real drafts should be used. If not speficied all
types of drafts will be used.

\item[\code{mflLeagues}] What type of leagues to include for MFL. Set to 0 to use
redraft leagues only; 1 to only use keeper leagues, 2 for rookie drafts, and
3 for MFL Public Leagues. If not speficied all types of drafts will be used.

\item[\code{ADPsource}] Character vector with one or more of
\code{"CBS", "ESPN", "FFC", "MFL", "NFL"}
\end{ldescription}
\end{Arguments}
\inputencoding{utf8}
\HeaderA{getADPdata}{Get ADP data}{getADPdata}
%
\begin{Description}\relax
Retrieve ADP data from multiple sources and combine into average.
\end{Description}
%
\begin{Usage}
\begin{verbatim}
getADPdata(ADPsources = c("CBS", "ESPN", "FFC", "MFL", "NFL"),
  season = as.POSIXlt(Sys.Date())$year + 1900, teams = 12,
  format = "standard", mflMocks = NULL, mflLeagues = NULL)
\end{verbatim}
\end{Usage}
%
\begin{Arguments}
\begin{ldescription}
\item[\code{season}] The season the ADP data is from

\item[\code{teams}] Number of teams in the league

\item[\code{format}] The format of the league, i.e. standard, ppr

\item[\code{mflMocks}] Include mock drafts from MFL. Set to 1 if only mock drafts
should be used, 0 if only real drafts should be used. If not speficied all
types of drafts will be used.

\item[\code{mflLeagues}] What type of leagues to include for MFL. Set to 0 to use
redraft leagues only; 1 to only use keeper leagues, 2 for rookie drafts, and
3 for MFL Public Leagues. If not speficied all types of drafts will be used.

\item[\code{ADPsource}] Character vector with one or more of
\code{"CBS", "ESPN", "FFC", "MFL", "NFL"}
\end{ldescription}
\end{Arguments}
\inputencoding{utf8}
\HeaderA{getCBSValues}{ADP data from CBS}{getCBSValues}
%
\begin{Description}\relax
Retrieve ADP data from CBS
\end{Description}
%
\begin{Usage}
\begin{verbatim}
getCBSValues()
\end{verbatim}
\end{Usage}
%
\begin{Value}
\LinkA{data.table}{data.table} with 3 columns:
\begin{description}

\item[cbsId] Player ID from CBS. Merge with results from \LinkA{getPlayerData}{getPlayerData}
to get player names
\item[adp] Average ADP
\item[leagueType] Assuming standard league type
\end{description}

\end{Value}
\inputencoding{utf8}
\HeaderA{getESPNValues}{ADP and auction value data from ESPN}{getESPNValues}
%
\begin{Description}\relax
Retrieve ADP and auction value from ESPN
\end{Description}
%
\begin{Usage}
\begin{verbatim}
getESPNValues()
\end{verbatim}
\end{Usage}
%
\begin{Value}
\LinkA{data.table}{data.table} with 6 columns:
\begin{description}

\item[player] Name of player
\item[position] Player position
\item[team] Team the player is playing for
\item[adp] Average ADP
\item[aav] Average auction value
\item[leagueType] Assuming standard league type
\end{description}

\end{Value}
\inputencoding{utf8}
\HeaderA{getFFCValues}{ADP data from FantasyFootballCalculator.com}{getFFCValues}
%
\begin{Description}\relax
Retrieve ADP data from fantasyfootballcalculator.com.
\end{Description}
%
\begin{Usage}
\begin{verbatim}
getFFCValues(format = "standard", teams = 12)
\end{verbatim}
\end{Usage}
%
\begin{Arguments}
\begin{ldescription}
\item[\code{format}] Format of league. Can be one of \code{"standard", "ppr", "2qb", "dynasty", "rookie"}.

\item[\code{teams}] Numer of teams in the league. One of 8, 10, 12, 14
\end{ldescription}
\end{Arguments}
%
\begin{Value}
\LinkA{data.table}{data.table} with 5 columns:
\begin{description}

\item[player] Name of player
\item[position] Player position
\item[team] Team the player is playing for
\item[adp] Average ADP
\item[leagueType] Chosen format for the league
\end{description}

\end{Value}
\inputencoding{utf8}
\HeaderA{getMeltedData}{Melt data into long form}{getMeltedData}
%
\begin{Description}\relax
Takes the data result with a projected stat in each column and melts the
data into one row per player per stat column.
\end{Description}
%
\begin{Usage}
\begin{verbatim}
getMeltedData(dataResult)
\end{verbatim}
\end{Usage}
%
\begin{Arguments}
\begin{ldescription}
\item[\code{dataResult}] A \LinkA{dataResult}{dataResult} object from the data scrape
\end{ldescription}
\end{Arguments}
\inputencoding{utf8}
\HeaderA{getMFLValues}{ADP and auction value data from MyFantasyLeague.com}{getMFLValues}
%
\begin{Description}\relax
Retrieve ADP and auction value data from MyFantasyLeague.com
\end{Description}
%
\begin{Usage}
\begin{verbatim}
getMFLValues(season = as.POSIXlt(Sys.Date())$year + 1900, type = "adp",
  teams = -1, ppr = -1, mock = -1, keeper = -1)
\end{verbatim}
\end{Usage}
%
\begin{Arguments}
\begin{ldescription}
\item[\code{season}] Year the data is retrieved for

\item[\code{type}] One of "adp" or "aav" to indicate whether ADP or auction values
should be retrieved.

\item[\code{teams}] Number of teams. If specified only drafts with that number of
teams will be include

\item[\code{ppr}] Specify if only ppr or non-ppr drafts should be considered. Set to 1
if only ppr drafts should be used, 0 if only standard drafts should be used.
If not specified all types of draft will be used.

\item[\code{mock}] Specify if only mock or real drafts should be used. Set to 1 if
only mock drafts should be used, 0 if only real drafts should be used. If not
speficied all types of drafts will be used.

\item[\code{keeper}] Specify to select what types of leagues should be used. Set to 0
to use redraft leagues only; 1 to only use keeper leagues, 2 for rookie drafts,
and 3 for MFL Public Leagues. If not  speficied all types of drafts will be used.
\end{ldescription}
\end{Arguments}
%
\begin{Value}
\LinkA{data.table}{data.table} wih up to 5 columns:
\begin{description}

\item[mflId] Player ID from MFL Merge with results from \LinkA{getPlayerData}{getPlayerData}
to get player names
\item[selectedIn] Number of drafts player has been selected in
\item[aav] Average auction value
\item[adp] ADP
\item[minPick] Earliest pick
\item[maxPick] Latest pick

\end{description}

\end{Value}
\inputencoding{utf8}
\HeaderA{getNFLValues}{ADP data from FantasyFootballCalculator.com}{getNFLValues}
%
\begin{Description}\relax
Retrieve ADP data from fantasyfootballcalculator.com.
\end{Description}
%
\begin{Usage}
\begin{verbatim}
getNFLValues()
\end{verbatim}
\end{Usage}
%
\begin{Value}
\LinkA{data.table}{data.table} with 5 columns:
\begin{description}

\item[esbid] Player ID for the player. Merge with results from \LinkA{getPlayerData}{getPlayerData}
to get player names
\item[player] Name of player
\item[position] Player position
\item[adp] Average ADP
\item[aav] Average auction value

\end{description}

\end{Value}
\inputencoding{utf8}
\HeaderA{getPlayerData}{Combine MFL and NFL player data}{getPlayerData}
%
\begin{Description}\relax
Retrieve data from MFL and NFL and combine
\end{Description}
%
\begin{Usage}
\begin{verbatim}
getPlayerData(season, weekNo, pos = position.name)
\end{verbatim}
\end{Usage}
%
\begin{Arguments}
\begin{ldescription}
\item[\code{season}] The year that player data is to be retrieved from

\item[\code{weekNo}] The weekNo that the player data is to be retrieved from

\item[\code{pos}] A character vector with position names to be retrieved
\end{ldescription}
\end{Arguments}
%
\begin{Value}
A \LinkA{data.table}{data.table} with 13 columns
\begin{description}

\item[playerId] NFL's ID for the player
\item[player] Name of player
\item[yahooId] Yahoo's ID for the player
\item[cbsId] CBS's ID for the player
\item[mflId] MFL's ID for the player
\item[position] The position the player is playing
\item[team] The team the player is playing for
\item[draftYear] The year the player was drafted
\item[birthData] The players birth date
\item[rookie] A logical value indicating whether the player is a rookie
\item[opponent] Team the player is facing next
\item[depthChart] Number on the depth chart for the player 1 = starter
\item[esbid] Alternate ID for player. Used for ADP/AAV data from NFL

\end{description}

\end{Value}
\inputencoding{utf8}
\HeaderA{getPlayerName}{Clean Player Data For Projections}{getPlayerName}
%
\begin{Description}\relax
For many of the the data soruces the player column contains more data than
needed to identify the player. With the help of regular expression data such as position,
team and injury information is cleaned from the player names.
\end{Description}
%
\begin{Usage}
\begin{verbatim}
getPlayerName(playerCol)
\end{verbatim}
\end{Usage}
%
\begin{Arguments}
\begin{ldescription}
\item[\code{playerCol}] The vector of player data taken from the data table that is returned from
the data scrape
\end{ldescription}
\end{Arguments}
%
\begin{Value}
The updated vector of player data
\end{Value}
\inputencoding{utf8}
\HeaderA{getProjections}{Calculate Projected Points}{getProjections}
%
\begin{Description}\relax
Calculate projected fantasy points, confidence intervals, risk, tiers, etc.
\end{Description}
%
\begin{Usage}
\begin{verbatim}
getProjections(scrapeData = NULL, avgMethod = "average",
  leagueScoring = scoringRules, vorBaseline, vorType, teams = 12,
  format = "standard", mflMocks = NULL, mflLeagues = NULL,
  adpSources = c("CBS", "ESPN", "FFC", "MFL", "NFL"), getADP = TRUE,
  getECR = TRUE, writeFile = TRUE)
\end{verbatim}
\end{Usage}
%
\begin{Arguments}
\begin{ldescription}
\item[\code{scrapeData}] The scraped projections data from \LinkA{runScrape}{runScrape}.

\item[\code{avgMethod}] A string specifying which average method to use for aggregating the
projections from different sources: mean ("average"), robust average ("robust"), or weighted average ("weighted"). Defaults to mean. Edit the analysts' weights for the weighted average in the \LinkA{analysts}{analysts} table.

\item[\code{leagueScoring}] List of scoring rules for the league see \LinkA{scoringRules}{scoringRules}
for an example.

\item[\code{vorBaseline}] The numbers (position rank values or point values) at each position to use for the baseline when
calculating VOR.

\item[\code{vorType}] Whether the baseline numbers are ranks or points. Defaults to position ranks.

\item[\code{teams}] Number of teams in the league (integer).

\item[\code{format}] League format ("standard" for standard leagues or "ppr" for Point-Per-Reception leagues).

\item[\code{mflMocks}] Whether to include mock drafts from MyFantasyLeague.com (MFL). Set to 1 to use only mock drafts,
0 to use only real drafts. If not specified, all draft types will be used.

\item[\code{mflLeagues}] What type of leagues to include for MyFantasyLeague.com (MFL). Set to 0 to use
only redraft leagues; 1 to use only keeper leagues, 2 for rookie drafts, and
3 for MFL Public Leagues. If not specified, all draft types will be used.

\item[\code{ADPsource}] Character vector with one or more of \code{c("CBS", "ESPN", "FFC", "MFL", "NFL")}.
\end{ldescription}
\end{Arguments}
%
\begin{Examples}
\begin{ExampleCode}
getProjections(scrapeData,                    ## Based on data in scrapeData
               avgMethod = "weighted",        ## calculate the projections using a weighted average
               leagueScoring = scoringRules,  ## using defined scoringRules,
               vorBaseline, vorType,          ## VOR Baselines and types
               teams = 12, format = "ppr",    ## for a 12 team ppr league
               mflMocks = 0, mflLeagues = 0,  ## using only real MFL redraft league
               adpSources =  c("FFC", "MFL")) ## and ADP data from MFL and FFC
\end{ExampleCode}
\end{Examples}
\inputencoding{utf8}
\HeaderA{getRanks}{Expert Consensus Rankings}{getRanks}
%
\begin{Description}\relax
Reterieve expert consensus rankings from fantasypros.com
\end{Description}
%
\begin{Usage}
\begin{verbatim}
getRanks(rank.position = "consensus", leagueType = "std", weekNo = 0)
\end{verbatim}
\end{Usage}
%
\begin{Arguments}
\begin{ldescription}
\item[\code{rank.position}] Position to retrieve ranks for. Use "consensus" to get
overall rankings

\item[\code{leagueType}] Indicate whether to get ppr rankings or standard rankings (std)

\item[\code{weekNo}] Week number to retrieve ranks for. Use 0 for season ranks.
\end{ldescription}
\end{Arguments}
%
\begin{Value}
\LinkA{data.table}{data.table} with up to 11 columns:
\begin{description}

\item[player] Name of player
\item[position] Player's position
\item[team] Team the player is playing for
\item[ecrRank] Consensus Rank
\item[bestRank] Highest rank from experts
\item[worstRank] Lowest rank from experts
\item[avgRank] Average rank from experts
\item[sdRank] Standard deviation of ranks
\item[adp] Average Draft Position
\item[vsAdp] Difference between overall rank and ADP
\item[rankType] Will have value of "overall" or "position"

\end{description}

\end{Value}
\inputencoding{utf8}
\HeaderA{getUrls}{URLs to scrape}{getUrls}
%
\begin{Description}\relax
Based on selected analysts, a given period and selected analysts generate
URLs that will be scraped. URLs are generated as merges of the \LinkA{analysts}{analysts},
\LinkA{sites}{sites}, \LinkA{siteUrls}{siteUrls}, and \LinkA{siteTables}{siteTables} datasets.
\end{Description}
%
\begin{Usage}
\begin{verbatim}
getUrls(selectAnalysts = analysts$analystId,
  period = periodType(dataPeriod()), positions = position.name)
\end{verbatim}
\end{Usage}
%
\begin{Arguments}
\begin{ldescription}
\item[\code{selectAnalysts}] An integer vector of selected analystIds.
See \LinkA{analysts}{analysts} for possible values

\item[\code{period}] A string indicating wether the URLs to produce are for weekly
or seasonal data scrape

\item[\code{positions}] A character vector of positions to scrape.
\end{ldescription}
\end{Arguments}
\inputencoding{utf8}
\HeaderA{getYahooValues}{ADP and auction value data from Yahoo}{getYahooValues}
%
\begin{Description}\relax
Retrieve ADP and auction value data from Yahoo
\end{Description}
%
\begin{Usage}
\begin{verbatim}
getYahooValues(type = "SD")
\end{verbatim}
\end{Usage}
%
\begin{Arguments}
\begin{ldescription}
\item[\code{type}] Draft type: AD for auction draft, SD for standard draft.
\end{ldescription}
\end{Arguments}
%
\begin{Value}
\LinkA{data.table}{data.table} with 3 columns:
\begin{description}

\item[cbsId] Player ID from CBS. Merge with results from \LinkA{getPlayerData}{getPlayerData}
to get player names
\item[adp/aav] Average ADP if type = "SD"; Average auction value if type = "AD"
\item[leagueType] Assuming standard league type
\end{description}

\end{Value}
\inputencoding{utf8}
\HeaderA{mflPlayers}{Read MFL Player Data}{mflPlayers}
%
\begin{Description}\relax
Function to read player data from MFL using the MFL API
\end{Description}
%
\begin{Usage}
\begin{verbatim}
mflPlayers(season = 2016, weekNo = 0, pos = position.name)
\end{verbatim}
\end{Usage}
%
\begin{Arguments}
\begin{ldescription}
\item[\code{season}] The year that player data is to be retrieved from

\item[\code{weekNo}] The weekNo that the player data is to be retrieved from

\item[\code{pos}] A character vector with position names to be retrieved
\end{ldescription}
\end{Arguments}
%
\begin{Value}
A \LinkA{data.table}{data.table} with 10 columns
\begin{description}

\item[playerId] NFL's ID for the player
\item[player] Name of player
\item[yahooId] Yahoo's ID for the player
\item[cbsId] CBS's ID for the player
\item[mflId] MFL's ID for the player
\item[position] The position the player is playing
\item[team] The team the player is playing for
\item[draftYear] The year the player was drafted
\item[birthData] The players birth date
\item[rookie] A logical value indicating whether the player is a rookie

\end{description}

\end{Value}
\inputencoding{utf8}
\HeaderA{nflPlayerData}{Player Data from NFL.com}{nflPlayerData}
%
\begin{Description}\relax
Retrieve player data from NFL.com
\end{Description}
%
\begin{Usage}
\begin{verbatim}
nflPlayerData(season = 2016, weekNo = 0, positions = position.name)
\end{verbatim}
\end{Usage}
%
\begin{Arguments}
\begin{ldescription}
\item[\code{season}] The year data is to be retrieved from

\item[\code{weekNo}] The week that data is to be retrieved from

\item[\code{positions}] A character vector of positions to be retrieved
\end{ldescription}
\end{Arguments}
%
\begin{Value}
A \LinkA{data.table}{data.table} with 7 columns:
\begin{description}

\item[playerId] NFL's ID For the player
\item[player] Name of player
\item[position] NFL's position designation for the player
\item[team] NFL Team that the player is playing for
\item[opponent] Team the player is facing next
\item[depthChart] Number on the depth chart for the player 1 = starter
\item[esbid] Alternate ID for player. Used for ADP/AAV data from NFL

\end{description}

\end{Value}
\inputencoding{utf8}
\HeaderA{periodType,dataPeriod-method}{Determine if period is a week or season}{periodType,dataPeriod.Rdash.method}
%
\begin{Description}\relax
Determine if period is a week or season
\end{Description}
%
\begin{Usage}
\begin{verbatim}
## S4 method for signature 'dataPeriod'
periodType(object)
\end{verbatim}
\end{Usage}
%
\begin{Arguments}
\begin{ldescription}
\item[\code{x}] A dataPeriod object
\end{ldescription}
\end{Arguments}
\inputencoding{utf8}
\HeaderA{projectPoints}{Calculate points}{projectPoints}
%
\begin{Description}\relax
For the scraped data, projected points, confidence interval, standard deviation
and position ranks are calculated
\end{Description}
%
\begin{Usage}
\begin{verbatim}
projectPoints(projectionData, scoringRules, avgType = "average")
\end{verbatim}
\end{Usage}
%
\begin{Arguments}
\begin{ldescription}
\item[\code{projectionData}] A \LinkA{data.table}{data.table} with projected stats

\item[\code{scoringRules}] A list with tables for league scoring rules. See
\LinkA{scoringRules}{scoringRules} for reference on format

\item[\code{avgType}] Which average to use. Should be one of \code{average, robust, weighted}
\end{ldescription}
\end{Arguments}
\inputencoding{utf8}
\HeaderA{readUrl}{Read data from a URL}{readUrl}
%
\begin{Description}\relax
The task for this function is to read the data from the URL location and assign
appropriate column names. The function will throw a warning if there are
more columns in the data table than expected. If there are fewer columns than
expected the function will retry up to 10 times to get the number of columns
correct. If it fails after 10 tries then an error will be thrown.
\end{Description}
%
\begin{Usage}
\begin{verbatim}
readUrl(inpUrl, columnTypes, columnNames, whichTable, removeRow, dataType,
  idVar, playerLinkString, fbgUser, fbgPwd)
\end{verbatim}
\end{Usage}
%
\begin{Arguments}
\begin{ldescription}
\item[\code{inpUrl}] The URL to get data from

\item[\code{columnTypes}] A character vector describing the types of columns in the
data. Note: \code{length(columnTypes) == length(columnNames)}.

\item[\code{columnNames}] A character vector describing the names of the columns in
the data. Note: \code{length(columnTypes) == length(columnNames)}.

\item[\code{whichTable}] A number or character describing the table to get. This
can be leveraged for HTML tables and spreadsheet files.

\item[\code{removeRow}] A numeric vector indicating rows to skip at the top of the data.
For exampe \code{c(1,2)} will skip the first two rows of data.

\item[\code{dataType}] A character indicating the type of data (HTML, XML, file, xls)
\end{ldescription}
\end{Arguments}
%
\begin{Value}
Returns a \LinkA{data.table}{data.table} with data from URL.
\end{Value}
\inputencoding{utf8}
\HeaderA{redistributeValues}{Redistribute values}{redistributeValues}
%
\begin{Description}\relax
Allows for the redistribution of values from \bold{one} variable to a set of
others based on the averages. For example, the function can be used to
redistribute total field goals to field goals per distances based of what
the average values are for each of the field goals per distance.
\end{Description}
%
\begin{Usage}
\begin{verbatim}
redistributeValues(valueTable = data.table(), calcType = "weighted",
  fromVar = "fg", toVars = c("fg0019", "fg2029", "fg3039", "fg4049",
  "fg50"), excludeAnalyst = 20)
\end{verbatim}
\end{Usage}
%
\begin{Arguments}
\begin{ldescription}
\item[\code{valueTable}] A \LinkA{data.table}{data.table}. Assumes outout from the
\LinkA{getMeltedData}{getMeltedData} function.

\item[\code{calcType}] A string specifying which calculation method to use for the
average values

\item[\code{fromVar}] A string specifying the name of the variable to distribute from

\item[\code{toVars}] A character vector with the names of variables to distribute to

\item[\code{excludeAnalyst}] An integer indicating an analyst to exclude. This will
exclude the analyst from the averages
\end{ldescription}
\end{Arguments}
\inputencoding{utf8}
\HeaderA{replaceMissingData}{Find replacement data for missing values}{replaceMissingData}
%
\begin{Description}\relax
For analysts that don't report on certain values the averages across other
analysts are calculated so they can be imputed.
\end{Description}
%
\begin{Usage}
\begin{verbatim}
replaceMissingData(statData = data.table(), calcType = "weighted")
\end{verbatim}
\end{Usage}
%
\begin{Arguments}
\begin{ldescription}
\item[\code{statData}] A \LinkA{data.table}{data.table}. Assumes outout from the
\LinkA{getMeltedData}{getMeltedData} function.

\item[\code{calcType}] A string specifying which calculation method to use for the
average values
\end{ldescription}
\end{Arguments}
\inputencoding{utf8}
\HeaderA{retrieveData}{Retrive data from a source table}{retrieveData}
%
\begin{Description}\relax
Data can be retrieved from a source table when specified along with a source
analyst and a data period. The data scrape will translate the data columns
for each source table into a uniform format.
\end{Description}
%
\begin{Usage}
\begin{verbatim}
retrieveData(srcTbl, srcPeriod, fbgUser = NULL, fbgPwd = NULL)
\end{verbatim}
\end{Usage}
%
\begin{Arguments}
\begin{ldescription}
\item[\code{srcTbl}] A \LinkA{sourceTable}{sourceTable} object representing the table to get data from

\item[\code{srcPeriod}] A \LinkA{dataPeriod}{dataPeriod} object representing the period to get

\item[\code{fbgUser}] User Name for an active footballguys.com account. Needed if
data scrape is requested from Footballguys

\item[\code{fbgPwd}] Password for an active footballguys.com account. Needed if
data scrape is requested from Footballguys

\item[\code{srcAnalyst}] A \LinkA{sourceAnalyst}{sourceAnalyst} object representing the analyst
to scrape data from
\end{ldescription}
\end{Arguments}
%
\begin{Value}
scrapeData a scrapeData object
\end{Value}
\inputencoding{utf8}
\HeaderA{runScrape}{Scrape Projections}{runScrape}
%
\begin{Description}\relax
Executes a scrape of players' fantasy football projections based on the selected
season, week, analysts, and positions. If no inputs are specified, the user is prompted.
\end{Description}
%
\begin{Usage}
\begin{verbatim}
runScrape(season = NULL, week = NULL, analysts = NULL, positions = NULL,
  fbgUser = NULL, fbgPwd, updatePlayers = TRUE)
\end{verbatim}
\end{Usage}
%
\begin{Arguments}
\begin{ldescription}
\item[\code{season}] The season of projections to scrape (e.g., 2015).

\item[\code{week}] The week number of projections to scrape (e.g., 16).
Week number should be an integer between 0 and 21.
Week number 0 reflects seasonal projections.
Week number between 1 and 17 reflects regular season projections.
Week number between 18 and 21 reflects playoff projections.

\item[\code{analysts}] An integer vector of analystIds specifying which analysts' projections to
scrape. See \LinkA{analysts}{analysts} data set for values of analystIds.

\item[\code{positions}] A character vector of position names specifying which positions
to scrape: \code{c("QB", "RB", "WR", "TE", "K", "DST", "DL", "LB", "DB")}.
\end{ldescription}
\end{Arguments}
%
\begin{Value}
list of \LinkA{dataResults}{dataResults}. One entry per position scraped.
\end{Value}
%
\begin{Note}\relax
The function has the ability to include subscription based sources,
but you will need to either download subscription projections separately or
provide a user name and password for those sites.
Scraping past seasons/weeks is nearly impossible because very few if any sites
make their historical projections available. An attempt to scrape historical
projections will likely produce current projections in most cases.
\end{Note}
%
\begin{Examples}
\begin{ExampleCode}
runScrape(season = 2016, week = 0,         ## Scrape 2016 season data for all
         analysts = 99, positions = "all") ## available analysts and positions

runScrape(season = 2016, week = 1,               ## Scrape 2016 week 1 data for
         analysts = c(-1, 5),                    ## CBS Average and NFL.com
         positions = c("QB", "RB", "WR", "TE",)) ## and offensive positions
\end{ExampleCode}
\end{Examples}
\inputencoding{utf8}
\HeaderA{Run\_Scrape}{Scrape gadget. Will be used as addin.}{Run.Rul.Scrape}
%
\begin{Description}\relax
Scrape gadget. Will be used as addin.
\end{Description}
%
\begin{Usage}
\begin{verbatim}
Run_Scrape()
\end{verbatim}
\end{Usage}
\inputencoding{utf8}
\HeaderA{scoringRules}{Default scoring rules.}{scoringRules}
\keyword{datasets}{scoringRules}
%
\begin{Description}\relax
The example below shows the default scoring rules implemented. The \code{ptsBracket}
element is only required if you have a \code{DST} element defined. To create a
custome scoring rule create a list with a data table for each position. Each
data table has two columns \code{dataCol, multiplier}. The \code{dataCol} column
is the name of the scoring variable and \code{multiplier} is the multiplier
to be used for the scoring variable. For example, in the default scoring rule
you can see that \code{passTds} for QB has a multiplier of 4 indicating that
4 points is awarded per passing TD.
\end{Description}
%
\begin{Usage}
\begin{verbatim}
scoringRules
\end{verbatim}
\end{Usage}
%
\begin{Examples}
\begin{ExampleCode}
scoringRules <- list(
   QB = data.table::data.table(dataCol = c("passYds", "passTds", "passInt", "rushYds", "rushTds", "twoPts", "fumbles"),
                               multiplier = c(1/25, 4, -3, 1/10, 6, 2, -3 )),
   RB = data.table::data.table(dataCol = c("rushYds", "rushTds", "rec", "recYds", "recTds", "returnTds", "twoPts", "fumbles"),
                               multiplier = c(1/10, 6, 0, 1/8, 6, 6, 2, -3)),
   WR = data.table::data.table(dataCol = c("rushYds", "rushTds", "rec", "recYds", "recTds", "returnTds", "twoPts", "fumbles"),
                               multiplier = c(1/10, 6, 0, 1/8, 6, 6, 2, -3)),
   TE = data.table::data.table(dataCol = c("rushYds", "rushTds", "rec", "recYds", "recTds", "returnTds", "twoPts", "fumbles"),
                               multiplier = c(1/10, 6, 0, 1/8, 6, 6, 2, -3)),
   K = data.table::data.table(dataCol = c("xp", "fg0019", "fg2029", "fg3039", "fg4049", "fg50"),
                              multiplier = c(1,  3, 3, 3, 4, 5)),
   DST = data.table::data.table(dataCol = c("dstFumlRec", "dstInt", "dstSafety", "dstSack", "dstTd", "dstBlk"),
                                multiplier = c(2, 2, 2, 1, 6, 1.5)),
   DL = data.table::data.table(dataCol = c("idpSolo", "idpAst", "idpSack", "idpInt", "idpFumlForce", "idpFumlRec", "idpPD", "idpTd", "idpSafety"),
                               multiplier = c(1, 0.5, 2, 3, 3, 2, 1, 6, 2)),
   LB =  data.table::data.table(dataCol = c("idpSolo", "idpAst", "idpSack", "idpInt", "idpFumlForce", "idpFumlRec", "idpPD", "idpTd", "idpSafety"),
                                multiplier = c(1, 0.5, 2, 3, 3, 2, 1, 6, 2)),
   DB = data.table::data.table(dataCol = c("idpSolo", "idpAst", "idpSack", "idpInt", "idpFumlForce", "idpFumlRec", "idpPD", "idpTd", "idpSafety"),
                               multiplier = c(1, 0.5, 2, 3, 3, 2, 1, 6, 2)),
   ptsBracket = data.table::data.table(threshold = c(0, 6, 20, 34, 99),
                                       points = c(10, 7, 4, 0, -4))
)
\end{ExampleCode}
\end{Examples}
\inputencoding{utf8}
\HeaderA{setTier}{Set tiers based on thresholds}{setTier}
%
\begin{Description}\relax
Set tiers based on thresholds
\end{Description}
%
\begin{Usage}
\begin{verbatim}
setTier(points, position)
\end{verbatim}
\end{Usage}
\inputencoding{utf8}
\HeaderA{sites}{Site data}{sites}
\keyword{datasets}{sites}
%
\begin{Description}\relax
Data table with information on sites that are being used for datascrapes.
\end{Description}
%
\begin{Usage}
\begin{verbatim}
sites
\end{verbatim}
\end{Usage}
%
\begin{Format}
A data.table with 17 rows and 5 columns
\begin{description}

\item[siteId] Unique integer identifier for the site
\item[siteName] The name of the site
\item[subscription] Indicator whether the site requires subscription to get
to the data
\item[playerId] Name of column in the player data table that holds the id
for the players

\end{description}
\end{Format}
\inputencoding{utf8}
\HeaderA{siteTables}{Table information}{siteTables}
\keyword{datasets}{siteTables}
%
\begin{Description}\relax
Data table with information on tables that data will be scraped from.
\end{Description}
%
\begin{Usage}
\begin{verbatim}
siteTables
\end{verbatim}
\end{Usage}
%
\begin{Format}
A \LinkA{data.table}{data.table} with 111 rows and 9 columns.
\begin{description}

\item[tableId] An integer that uniquely identifies the table
\item[position] Name of the position that the table holds data for
\item[positionAlias] The position identifier used in the URL for the table
\item[siteId] Integer identfying the site that the table is found on. See
\LinkA{sites}{sites} for values
\item[startPage] If the table spans more pages then the start number for the
pages. If all the data is on one page the value is 1.
\item[endPage] If the table spans more pages then the end number for the page
sequence. If all the data is on one page the value is 1
\item[stepPage] If the table spans more page then the step number of the page
sequence. If all the data is on one page the value is 1
\item[season] Indicates whether the table can be used for seasonal data scrapes.
\item[weekly] Indicates whether the table can be used for weekly data scrapes.

\end{description}
\end{Format}
\inputencoding{utf8}
\HeaderA{siteUrls}{URL information}{siteUrls}
\keyword{datasets}{siteUrls}
%
\begin{Description}\relax
Data table with information on the URLs that data will be scraped from.
\end{Description}
%
\begin{Usage}
\begin{verbatim}
siteUrls
\end{verbatim}
\end{Usage}
%
\begin{Format}
A \LinkA{data.table}{data.table} with 31 rows and 7 columns.
\begin{description}

\item[siteId] An integer identifying the site the URL is referring to. See
\LinkA{sites}{sites} for values.
\item[siteUrl] The URL that the data will be scraped from. Use placeholders
for parameters that needs to be substituted, like position, analyst, page,
season and week.
\item[urlPeriod] Specifies whether the URL is used for week or season data.
\item[urlType] Specifies the data type that is returned from the URL. Could be
HTML, XML, csv or file
\item[urlTable] The table number that the data scrape should read from the
URL. Mostly used for HTML data, but can also be used to designate sheets in
a spreadsheet file.
\item[playerLink] A string to identify the links to player profile pages on
an HTML page.

\end{description}
\end{Format}
\inputencoding{utf8}
\HeaderA{sourceAnalyst-class}{Class to represent the source Analysts}{sourceAnalyst.Rdash.class}
\aliasA{sourceAnalyst}{sourceAnalyst-class}{sourceAnalyst}
%
\begin{Description}\relax
Class to represent the source Analysts
\end{Description}
%
\begin{Section}{Slots}

\begin{description}

\item[\code{analystName}] The name of the analysts

\item[\code{sourceId}] The id for the source. This is only used if there are multiple
analysts for a site

\item[\code{analystId}] A character string specifying the id to be used for the analyst.
If left blank, it will be set as a 4 letter abbreviation of the site + analyst
names using \LinkA{abbreviate}{abbreviate}.

\end{description}
\end{Section}
%
\begin{Examples}
\begin{ExampleCode}
cbs.avg <- sourceAnalyst(analystName = "Average",
                         sourceId = "avg",
                         analystId = "cbav")
\end{ExampleCode}
\end{Examples}
\inputencoding{utf8}
\HeaderA{sourceSite-class}{Class to represent source sites}{sourceSite.Rdash.class}
\aliasA{sourceSite}{sourceSite-class}{sourceSite}
%
\begin{Description}\relax
Source sites are one of the foundations to the data scrapes. They are
representing the web sites that provides data for the projections. The
sources can be different for weekly and seasonal data and the type of
data can be different. If the data is scraped from HTML then the playerId
can be derived as well.
\end{Description}
%
\begin{Section}{Slots}

\begin{description}

\item[\code{siteId}] The ID for the site as identfied in configuration data

\item[\code{siteName}] Name of source site

\item[\code{siteUrl}] String that represents the URL that the site uses for the seasonal data, if any.

\item[\code{urlType}] String that identifies the type of data in  seasonURL (HTML, XML, CSV or file)

\item[\code{urlTable}] A character string identifying which table to grab from seasonURL

\item[\code{playerLink}] A string representing part of the URL to a player profile

\item[\code{playerId}] What to call the id number for player if present

\end{description}
\end{Section}
%
\begin{Note}\relax
When specifying the URLs paramters can be used as place holders.
\end{Note}
%
\begin{Examples}
\begin{ExampleCode}
sourceSite(siteName = "CBS",
           siteUrl = "http://www.cbssports.com/fantasy/football/stats/weeklyprojections/{$Pos}/season/{$SrcID}/standard?&print_rows=9999",
           urlType = "html",
           urlTable = "1",
           playerLink = "/fantasyfootball/players/playerpage/[0-9]{3,6}",
           playerId = "cbsId")
\end{ExampleCode}
\end{Examples}
\inputencoding{utf8}
\HeaderA{sourceTable-class}{Class to represent a data table from a source site}{sourceTable.Rdash.class}
\aliasA{sourceTable}{sourceTable-class}{sourceTable}
%
\begin{Description}\relax
Source table class extends the \LinkA{sourceAnalyst}{sourceAnalyst} class
\end{Description}
%
\begin{Section}{Slots}

\begin{description}

\item[\code{sourcePosition}] The position designation that the table represents

\item[\code{positionAlias}] The designation that the source site uses for the position

\item[\code{startPage}] If the table covers multiple pages, the start numbering for the
pages. Otherwise 1

\item[\code{endPage}] If the table covers multiple pages, the end numbering for the
pages. Otherwise 1

\item[\code{stepPage}] If the table covers multiple pages, the step in the sequence of
page numbering. Otherwise 1

\item[\code{tableId}] ID from configuration data that identifies the table uniquely

\end{description}
\end{Section}
%
\begin{Examples}
\begin{ExampleCode}
cbs.qb <- sourceTable(sourcePosition = "QB",
                      positionAlias = "QB",
                      startPage = 1,
                      endPage = 1,
                      stepPage = 1,
                      tableId = 1
                     )
\end{ExampleCode}
\end{Examples}
\inputencoding{utf8}
\HeaderA{tableColumns}{Constants defining result columns}{tableColumns}
\aliasA{resultColumns}{tableColumns}{resultColumns}
\aliasA{staticColumns}{tableColumns}{staticColumns}
\keyword{datasets}{tableColumns}
%
\begin{Description}\relax
The results in the \LinkA{dataResult}{dataResult.Rdash.class} table are determined by the values in
these constants. See definitiions in details.
\begin{description}

\item[staticColumns] Names of columns in all result tables.
\item[resultColums] A list of character vectors (one per position), indicating
the columns in the results table for that position.

\end{description}


Data table with information on the columns in tables identified in \LinkA{siteTables}{siteTables}.
\end{Description}
%
\begin{Usage}
\begin{verbatim}
staticColumns

resultColumns

tableColumns
\end{verbatim}
\end{Usage}
%
\begin{Format}
A \LinkA{data.table}{data.table} with 2924 rows and 6 columns.
\begin{description}

\item[tableId] An integer identifying the table that the column belongs to.
See \LinkA{siteTables}{siteTables} for values
\item[columnName] The name of the column
\item[columnType] The data type for the column
\item[columnOrder] The order in which the column appears in the table
\item[columnPeriod] Indicates if the column appears in tables used for seasonal
or weekly data scrapes
\item[removeColumn] Indicates if the column can be removed before the table
is returned.

\end{description}
\end{Format}
\inputencoding{utf8}
\HeaderA{tableRowRemove}{Table rows to remove}{tableRowRemove}
\keyword{datasets}{tableRowRemove}
%
\begin{Description}\relax
Data table with information on rows that will need to be removed from \LinkA{siteTables}{siteTables}
for the data scrape to be succesfull
\end{Description}
%
\begin{Usage}
\begin{verbatim}
tableRowRemove
\end{verbatim}
\end{Usage}
%
\begin{Format}
A \LinkA{data.table}{data.table} with 28 rows and 2 columns.
\begin{description}

\item[tableId] An integer identifying the table that the row can be removed from.
See \LinkA{siteTables}{siteTables} for values
\item[rowRemove] An integer identifying the row that can be removed

\end{description}
\end{Format}
\printindex{}
\end{document}
